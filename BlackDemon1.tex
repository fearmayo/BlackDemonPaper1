\documentclass[twocolumn,iop,revtex4]{openjournal}
\usepackage{natbib} 
\usepackage[dvipsnames]{xcolor}
\usepackage{aas_macros} 
\usepackage{amssymb}
\usepackage{amsmath}
\usepackage[title]{appendix}
\usepackage{hyperref}	% Hyperlinks
\hypersetup{colorlinks=true,linkcolor=blue,citecolor=blue,filecolor=blue,urlcolor=blue}
\usepackage[caption=false]{subfig}
\usepackage{calc}
%\usepackage{soul}                          % provides \hl{} for highlighting
\usepackage{ulem} \normalem 
\usepackage{xifthen}
\newcommand{\lya}{Lyman-$\alpha$~}
\newcommand{\gad}{\textsc{Gadget-2~}}
\newcommand{\enzo}{\texttt{Enzo~}}
\newcommand{\enzoc}{\texttt{Enzo}}
\newcommand{\yt}{\texttt{yt~}}
\newcommand{\ytc}{\texttt{yt}}
\newcommand{\cloudy}{\texttt{Cloudy~}}
\newcommand{\grackle}{\texttt{Grackle-2.1~}}
\newcommand{\gracklec}{\texttt{Grackle-2.1}}
\newcommand{\enzolat}{\texttt{Enzo-2.3}}
\newcommand{\cosmos}{\textsc{Cosmos~}}

\newcommand{\darwin}{\textsc{Darwin~}}
\newcommand{\enzoamr}{\texttt{Enzo(AMR)~}}
\newcommand{\enzost}{\texttt{Enzo(static)~}}
\newcommand{\lyb}{Lyman-$\beta$~} 
\newcommand{\eg}{{\it e.g.~}}
\newcommand{\kms} {km $\rm{s^{-1}}$}
\newcommand{\mpch} {\rm $h^{-1}$ Mpc\,\,} 
\newcommand{\kpch} {\rm $h^{-1}$ kpc\,\,} 
\newcommand{\msolar} {$\rm{M_{\odot}}~$}
\newcommand{\msolarc} {$\rm{M_{\odot}}$}
%\newcommand{\msolaryr} {$\rm{M_{\odot}/yr}~$}
%\newcommand{\msolaryrc} {$\rm{M_{\odot}/yr}$}
\newcommand{\msolaryr} {$\rm{M_{\odot}~yr^{-1}}~$}
\newcommand{\msolaryrc} {$\rm{M_{\odot}~yr^{-1}}$}
\newcommand{\lsolar} {$\rm{L_{\odot}}~$}
\newcommand{\lsolarc} {$\rm{L_{\odot}}$}
\newcommand{\zsolar} {$\rm{Z_{\odot}}~$}
\newcommand{\zsolarc} {$\rm{Z_{\odot}}$}
\newcommand{\rsolar} {$\rm{R_{\odot}}~$}
\newcommand{\molH} {$\rm{H_2}$~}
\newcommand{\molHc} {$\rm{H_2}$}
\newcommand{\J} {$\rm{10^{-21}\ erg\ cm^{-2}\ s^{-1}\ Hz^{-1}\ sr^{-1}}$}
\newcommand{\inten} {$\rm{ erg\ cm^{-2}\ s^{-1}\ Hz^{-1}\ sr^{-1}}$~}
\newcommand{\JU} {$\rm{ erg\ cm^{-2}\ s^{-1}\ Hz^{-1}\ sr^{-1}}$}
\newcommand{\JLW} {J$_{\rm LW}$}
\newcommand{\healpix} {\texttt{HEALPix~}}
\newcommand{\smartstar} {\texttt{SmartStar~}}
\newcommand{\smartstars} {\texttt{SmartStars~}}
\newcommand{\smartstarsc} {\texttt{SmartStars}}
\newcommand{\smartstarc} {\texttt{SmartStar}}
\newcommand{\rarepeak} {\textit{Rarepeak~}}
\newcommand{\rarepeakc} {\textit{Rarepeak}}
\newcommand{\normal} {\textit{Normal~}}
\newcommand{\normalc} {\textit{Normal}}
\newcommand{\void} {\textit{Void~}}
\newcommand{\voidc} {\textit{Void}}
\newcommand{\ha} {\texttt{HaloA~}}
\newcommand{\hb} {\texttt{HaloB~}}
\newcommand{\hac} {\texttt{HaloA}}
\newcommand{\hbc} {\texttt{HaloB}}

\def\mgh#1{{\bf MH:  #1}}
\def\jr#1{{\color{blue} \bf JR:  #1}}
\def\pj#1{{\bf PJ:  #1}}
\newcommand{\jhw}[1]{{\color{Maroon} (\bf JHW: #1)}}
\newcommand{\note}[1]{{\noindent\hspace{-3em}\bf\color{Plum}$\longrightarrow$ \quad{#1}}}
\newcommand{\delete}[1]{{\color{red}{\sout{#1}}}}
\newcommand{\change}[2][]{%
\ifthenelse{\isempty{#2}}{{\color{ForestGreen}{#1}}}%
{{\color{RedOrange}\sout{#1}}{\color{ForestGreen}{ #2}}}%
}


\def\etal{{\it et al.}~}

\begin{document}
\title[The First Stars were Super Massive]{Massive Star Formation in Overdense Regions of the Early Universe}
\author{John A. Regan$^{1,2,*}$}
\thanks{$^*$E-mail:john.regan@mu.ie, Royal Society - SFI University Research Fellow}
%\author{John H. Wise$^{3}$}
%\author{Tyrone E. Woods$^{4}$}
%\author{Turlough P. Downes$^{2}$}
%\author{Brian W. O'Shea$^{5,6,7, 8}$}
%\author{Michael L. Norman$^9$}


\affiliation{$^1$Department of Theoretical Physics, Maynooth University, Maynooth, Ireland}
%\affiliation{$^2$Centre for Astrophysics \& Relativity, School of Mathematical Sciences, Dublin City University, Glasnevin, D09 W6Y4, Ireland}
%\affiliation{$^3$Center for Relativistic Astrophysics, Georgia Institute of Technology, 837 State Street, Atlanta, GA 30332, USA}
%\affiliation{$^4$National Research Council of Canada, Herzberg Astronomy \& Astrophysics Research Centre, 5071 West Saanich Road, Victoria, BC V9E 2E7, Canada}
%\affiliation{$^5$Department of Computational Mathematics, Science, and Engineering, Michigan State University, MI, 48823, USA}
%\affiliation{$^6$Department of Physics and Astronomy, Michigan State University,MI, 48823, USA}
%\affiliation{$^7$Joint Institute for Nuclear Astrophysics - Center for the Evolution of the Elements, USA}
%\affiliation{$^8$National Superconducting Cyclotron Laboratory, Michigan State, University, MI, 48823, USA}
%\affiliation{$^9$Center for Astrophysics and Space Sciences, University of California, San Diego, 9500 Gilman Dr, La Jolla, CA 92093}

%\pubyear{2020}
%\label{firstpage}
%\pagerange{\pageref{firstpage}--\pageref{lastpage}}
%\maketitle

\begin{abstract}
  skdofjkdlsakl
\end{abstract}

\keywords{Early Universe, Supermassive Stars, Star Formation, First Galaxies, Numerical Methods}

\section{Introduction} \label{Sec:Introduction}
Supermassive stars (SMSs) with masses between $10^4$ and $10^5$ \msolar have, over the past few decades, been invoked \citep{Rees_1978, Begelman_1978, Begelman_2006,
  Begelman_2008, Latif_2016a, Woods_2018} as an intermediate phase to explain the existence of supermassive black holes (SMBHs) at the centres of massive galaxies \citep{Fan_06, Kormendy_2013}.
The pathways to forming a SMBH are thus far unknown with a number of theoretical models proposing to explain their existence. \\
\indent Perhaps the
simplest explanation is to invoke the black holes left over from the first 
generation of stars as seeds for SMBHs. The first generation of (metal-free) 
stars are referred to as Population III (PopIII) stars and according to 
current theoretical models \citep[e.g.][]{Turk_2009, Clark_2008, Hirano_2014, Stacy_2016} the initial mass function should be
top heavy with a characteristic mass of tens of solar masses. However, PopIII remnant black holes are expected to form in low
density environments \citep{Whalen_2004, OShea_2005b, Milosavljevic_2009} and are not expected to accrete substantially, at least not 
initially \citep{Alvarez_2009, Smith_2018}. PopIII stars are therefore not seen as good candidates to explain the existence
of SMBHs without invoking super-Eddington accretion scenarios which can boost their initial 
seed masses by an order of magnitude or more over a short period
\citep{Lupi_2014, Pacucci_2015a, Sakurai_2016a,Inayoshi_2016, Pacucci_2017, Inayoshi_2018}.\\
\indent SMSs provide an alternative path to forming a SMBH by giving the seed black hole a
head-start compared to a black hole formed from a PopIII remnant. Under a SMS formation scenario
the accretion rate onto the protostar must exceed a critical threshold thought to be around
0.001 \msolaryr \citep{Haemmerle_2017}. When
this threshold accretion rate is reached and maintained the stellar radius inflates reducing
its surface temperature to approximately 5000 K
and making the star resemble a red giant star
\citep{Omukai_2003, Hosokawa_2012, Hosokawa_2013, Woods_2017}. However, the SMS must continue to
accrete above this threshold rate. If the accretion rate falls below the critical
rate, for a time exceeding the Kelvin-Helmholtz time \citep{Sakurai_2016}, the star contracts to the main sequence and becomes a hyper-luminous PopIII
star with a mass set approximately by the mass at which the accretion rate dropped.
When the accretion rate is maintained the star grows rapidly but emits
only weak radiative feedback with the spectrum of the emitted radiation peaking below
the hydrogen ionisation limit \citep{Woods_2018}. \\
\indent As discussed, the key requirement for
forming a SMS is that the mass accretion rate onto the star exceeds approximately 0.001 \msolaryrc,
however, a sufficient baryon reservoir is also required and furthermore the metallicity of the gas
being accreted should be below $10^{-3}$ \zsolarc.  Detailed high resolution simulations have found that gas that has been enriched above this threshold fragments into lower mass stars which do not converge to a single object and in this case the formation of a SMS is suppressed \citep[e.g.][]{Chon_2020}. For these reasons
metal-poor (i.e., Z $\lesssim 10^{-3}$ \zsolarc) atomic cooling haloes are seen as the most 
promising candidates in which to form SMSs. Haloes which have higher levels of 
metal-enrichment (Z $> 10^{-3}$ \zsolarc) may also be viable candidates for 
SMS formation in the early universe if metal mixing is inhomogeneous \citep{Regan_2020a}. Atomic cooling
haloes which provide the above requirements for SMS formation were recently investigated by \cite{Wise_2019}
and \cite{Regan_2020}. In particular \cite{Wise_2019} found that the combination 
of a mild Lyman-Werner background combined with the impact of dynamical heating effects due to
minor and major mergers can suppress star formation until a halo crosses the atomic cooling threshold.
These haloes are therefore predominantly metal-poor (with any metal enrichment coming externally), have large 
baryon reservoirs and suppressed \molH content due to the LW radiation fields. 
In this study we build on the previous works cited above. \\
\indent The goal of this study is to model the formation and evolution of (super-)massive star
formation in haloes that are exposed to moderate LW backgrounds, which when combined with
the effects of dynamical heating can suppresses star formation below the atomic cooling limit.
  To pursue this research we re-simulate
  two haloes from the original Renaissance simulations using the zoom technique.
  We designate
  these haloes as \ha and \hbc. Both haloes were chosen as they exhibited near isothermal
  collapse of their inner core in the original Renaissance datasets as shown by \cite{Regan_2020}.
  They were therefore identified as among the most promising candidates for SMS formation. Both
  haloes were exposed to moderate levels of LW radiation from nearby radiation sources as well as
  constant mergers which dynamically heated the gas within the haloes. To further understand the
  impact of the LW field we re-simulate \hb with no LW radiation in this work. This is done to
  determine if a halo can remain star-free due to only dynamical heating effects or if the LW field 
  remains a critical component. \hac, on the other-hand, is re-simulated with a LW background
  composed of both local source contributions and background contributions. \\
  \indent In the zoom simulations, we find that \ha forms stars with masses greater than
  6000 \msolar but that the accretion rate onto individual proto-stars always declines as the star's
  immediate gas supply is depleted. In the re-simulation of \hbc, without a LW field, we find that
  the halo undergoes premature collapse (compared to the original case where a LW field of
  \JLW $\sim 2$ J$_{21}$\footnote{J$_{21}$\ is shorthand for $1 \times 10^{-21} \ \rm{ erg\ cm^{-2}\ s^{-1}\ Hz^{-1}\ sr^{-1}}$} existed). In \hb the most massive star in the halo has a mass of
  approximately 173 \msolarc. The impact of ionising radiation is not considered in these
  simulations but post-processing of the stellar feedback using \cloudy \citep{Ferland_2017} is
  instead used to gauge the likely impact of ionising sources, particularly for \ha which forms a
  number of hyper-luminous PopIII stars. \\
  \indent The paper is laid out as follows: In \S \ref{Sec:Methods} we very briefly review the
  original Renaissance simulations as well as discussing the zoom-in simulations. In \S
  \ref{Sec:Results} we analyse the results of the zoom-in simulations. In \S \ref{Sec:GW} we
  discuss the implications of the results and the connection with upcoming gravitational wave
  observatories. In \S \ref{Sec:Discussion} we summarize our results and outline
  our conclusions.

\section{Methods} \label{Sec:Methods}


\subsection{Black Demon Simulation Suite} \label{Sec:BlackDemon}
\enzo has been extensively used to study the formation of structure in the early universe
\citep{Abel_2002, OShea_2005b, Turk_2012, Wise_2012b, Wise_2014, Regan_2015, Regan_2017}.
\enzo includes a ray tracing scheme to follow the propagation of radiation from
star formation and black hole formation \citep{WiseAbel_2011} as well as a detailed multi-species
chemistry model that tracks the formation and evolution of nine species \citep{Anninos_1997,
  Abel_1997}. In particular the photo-dissociation of \molH is followed, which is a critical
ingredient for determining the formation of the first metal-free stars \citep{Abel_2000}.\\
\indent The original Renaissance simulations \cite{Xu_2013, Xu_2014, OShea_2015} were carried out
on the Blue Waters supercomputer using the adaptive mesh refinement
code \enzo\citep{Enzo_2014, Enzo_2019}\footnote{https://enzo-project.org/}.
The datasets that formed the basis for this study were originally derived from a simulation of the
universe in a 40 Mpc on the side box using the WMAP7 best fit cosmology \citep{Komatsu_2011}.
For more details on the Renaissance simulation suite see \cite{Chen_2014}. Here we outline only
the details relevant to this study for brevity. The simulation suite was broken down into
three separate regions, namely the \rarepeakc, \normal and \void regions. Each region was simulated
with an effective initial resolution of $4096^3$ grid cells and particles giving a maximum dark matter
particle mass resolution of $2.9 \times 10^4$ \msolarc. Further refinement was allowed throughout
each region up to a maximum refinement level of 12, which corresponded to 19 pc comoving spatial
resolution. Given that the regions focus on different
 overdensities each region was evolved forward in time to different epochs. The \rarepeak region,
 being the most overdense and hence the most computationally demanding at earlier times, was run
 until $z = 15$. The \normal region ran until $z = 11.6$, and the \void region ran until $z = 8$.
 In all of the regions the halo mass function was very well resolved down to M$_{\rm halo} \sim 2
 \times 10^6$ \msolarc. \\
 \indent  As noted already in \S \ref{Sec:Introduction}, in \cite{Wise_2019} we examined two
 metal-free and star-free haloes from the final output of the \rarepeak simulation and re-simulated
 those two haloes at significantly higher resolution (maximum spatial resolution,
 $\Delta x \sim 60$ au) until the point of collapse. This re-simulation allowed us to investigate
 the evolution of the inner halo and the mass distribution of the clumps formed. However, no star
 formation prescription was employed during this re-simulation. In \cite{Regan_2020} we subsequently
 investigated the occurrence of metal-free and star-free atomic cooling haloes across all of the
 Renaissance datasets. We found a total of 79 such haloes in the
 \rarepeak outputs and three such haloes in the \normal outputs. None were found in the \void outputs.
 Of the 79 haloes which were metal-free and star free above the atomic cooling limit four haloes showed almost ideal isothermal collapse (see Figure 6. in \cite{Regan_2020}) consistent with what has previously been identified
 as ideal conditions for forming SMSs \citep{Inayoshi_2014, Becerra_2015, Latif_2016a,
   Regan_2017, Chon_2017b, Regan_2018b}. Of those haloes which collapsed isothermally we then
 selected two haloes for re-simulation in this study. \\


 \subsection{Subgrid Star Formation Prescription} \label{Sec:StarFormation}
 \indent In order to resolve star formation in the collapsing target haloes we set the maximum refinement
 level of the simulation to 20. This is an increase of a factor of $2^8 (256)$ compared to the
 original Renaissance simulations and allows us to reach a maximum spatial resolution of $\Delta x \sim 1000$~au.
 While this (maximum) resolution is less than what was achieved
 in \cite{Wise_2019} it was necessary as the goal of this re-simulation was not only to follow the
 collapse of the target halo but to also follow the formation of stars within the collapsing
 halo for up to 2 Myr following the formation of the first star. At the resolution used in
 \cite{Wise_2019} this proved intractable and so we reduced the
 resolution by a factor of $2^4 (16)$, compared to \cite{Wise_2019}, as a compromise. Reducing the
 refinement factor compared to \cite{Wise_2019} reduced the computational load while still allowing us to
 resolve star formation at an acceptable resolution. \\
 \indent In order to model star formation within the collapsing gas cloud we employed a star
 formation criteria using the methodology first described in \cite{Krumholz_2004}. The implementation
 in \enzo is described in detail in \cite{Regan_2018a} and \cite{Regan_2018b} and we give a
 brief overview here for completeness. Stars are formed when all of the following conditions are met:
\begin{enumerate}
\item The cell is at the highest refinement level
\item The cell exceeds the Jeans density 
\item The flow around the cell is converging 
\item The cooling time of the cell is less than the freefall time
\item The cell is at a local minimum of the gravitational potential
\end{enumerate}
Once the star is formed accretion onto the star is determined by evaluating the mass flux across a
sphere with a radius of 4 cells centered on the star. Initially all stars are assumed to be stars with low surface
temperatures that are appropriate for main sequence SMSs and less massive proto-stars on the
Hayashi track. The accretion onto the surface of the embryonic star is found by applying Gauss's
divergence theorem to the volume integral of the accretion zone \citep[e.g][]{Bleuler_2014}
(i.e. the volume integral of flux inside the accretion zone)
\begin{equation}
  \dot{M} = 4\pi \int_\Omega { \rho v_r^- r^2 dr}
\end{equation}
where $\dot{M}$ is the mass accretion rate, $\Omega$ is the accretion zone over which we integrate,
$\rho$ is the
density of the cells intersecting the surface, $v_r^-$ is the velocity of cells intersecting
the surface which have negative radial velocities and $r$ is the radius of our surface. As noted above we
set the accretion radius to be 4 cells. The accretion onto the star is calculated at each timestep,
however this is likely to be a very noisy metric. To alleviate this to some degree we average
the accretion rate over intervals of 1 kyr and use that averaged accretion rate in data outputs.
The accretion rate is added as an attribute to each star and hence a full
accretion history of every star is outputted as part of every snapshot. Mergers with other stars
are also included in the accretion onto the stars. Simulations of massive star formation
\citep[e.g.][]{Meyer_2020} do show that stellar mergers do increase the “bloatedness” of a star
and hence including the mergers in the accretion rates is likely to be valid.
In this case the more massive star retains its
information (e.g., age, type, etc.) after the merger event - information on the less massive star is
lost. The mass of the less massive star is added to the accretion rate of the more massive star for
that timestep. Stars are merged when they come within 3 times the accretion radius of each other (i.e. 12 cell lengths).
We also note here that the dark matter particle mass is approximately $M_{\rm DM} \sim 170$ \msolarc.
  This mass is comparable to and greater than the mass of some of the stars that are formed. While
  the star forming regions are strongly baryon dominated individual dark matter particles may produce
  some shot noise. The shot noise effects are accounted for however by smoothing the dark matter particles
  over a scale of approximately 1 pc. This alleviates the impact of individual dark matter particles
  on the central gas and star particles \citep[e.g.][]{Regan_2015}. \\

\indent Each star also has the ability to provide both radiative and mechanical feedback, which is
most appropriate in the case where the star has transitioned into a (accreting) black hole. As the
accretion rate onto the star varies the star can transition its type from a SMS, with an inflated
surface, to a PopIII star.  This transitioning only occurs if the accretion rate onto the star either
never exceeds the critical rate (set in our simulations be be 0.04 \msolaryrc) or if the accretion
rate onto the star falls below the critical accretion rate. While the star remains bloated the
radiative
feedback from the star is primarily below the hydrogen ionisation limit and is mostly in the
form of infrared radiation. However, if the accretion rate drops and the star contracts to the
main sequence then its surface temperature dramatically increases up to $10^5$~K,
causing its spectrum to harden and peak in the UV. \\
\indent All stars in this simulation emit radiative feedback below the ionisation threshold of
hydrogen. The radiation is followed explicitly using the ray tracing technique \citep{WiseAbel_2011}.
Pop III stars are modelled assuming a blackbody spectrum with a characteristic mass of 40 \msolarc
 (Table 4, \cite{Schaerer_2002}). From that we assign a \textit{LuminosityPerSolarMass} to the
Pop III star and the star consequently becomes more luminous and the feedback more intense as the mass
of the star increases.  SMSs are modelled by assuming
a blackbody spectrum with an effective temperature of T$_{\rm eff}$ = 5500 K \citep{Hosokawa_2013}.
The radiation spectrum for a SMS therefore peaks in the infrared as opposed to the UV for Pop
III stars. For the specific luminosity of the SMS we take a characteristic mass of 500 \msolar and
apply the contribution from the non-ionising photons only \citep{Schaerer_2002}. As with the 'normal'
Pop III stars the SMS luminosity changes as mass is accreted and the total luminosity then
scales up as the mass increases. \\
\indent In both cases the radiation from the stars is propagated outwards from the star using the
\texttt{MORAY} radiative transfer package \citep{WiseAbel_2011} that is part of \enzoc.
\texttt{MORAY} is able to model the ionisation of H, He and He$^{+}$. It can also account for the
photo-dissociation of \molH for photons with energies within the Lyman-Werner band and the
photo-detachment of  $\rm{H^-}$ and $\rm{H_2^+}$ for photons in the infrared band. For each type of
star we use five energy bins. The first two energy bins (E $< 13.6$ eV) are weighted by the cross
section peaks for $\rm{H^-}$,  $\rm{H_2^+}$ and \molH photo detachment/dissociation respectively.
The next three energy bins are determined using the \texttt{sedop} code developed by
\cite{Mirocha_2012} which determines the optimum number of energy bins needed to
accurately model radiation with energy above the ionisation threshold of hydrogen. For the
self-shielding of \molH against LW radiation we use the prescription of \cite{Wolcott-Green_2011}.\\
\indent As stars in this simulation contract to the main sequence, when the accretion rate onto the
star drops, radiation above the hydrogen ionisation threshold is not initiated although in
principle it should be and our radiative transfer scheme does support ionising radiation. The
problem however is that the shocks generated near
the star particle are too strong and unresolved for the PPM reconstruction scheme and HLLC Riemann
solver to handle, the steep gradients cause the solution to overshoot to negative densities and
energies \citep[see][for details]{Enzo_2014}. We could have avoided this problem by utilizing a more
diffusive solver at the expense of accuracy, but instead
we neglect the impact of ionising feedback and post-process the impact of
ionising feedback with \cloudy and use the results to estimate the extent to which the stars
(particularly the hyper luminous stars in \hac) ionise their surroundings and potentially shutdown
further star formation\footnote{Initially our simulation plan had been to run four haloes with
  and without a background LW field at a refinement level of 24 (16 times higher than we report
  here). However, the simulations proved intractable and computationally expensive and hence we
  reduced the scope of the campaign. As a result, we did not have the capacity to run
    \ha without a LW radiation field or \hb with a radiation field.
  We are now in the process of undertaking a new
  more expansive campaign to address these shortcomings.}.

%%%%%%%%%%%%%%%%%%%%%%%%%%%%%%%%%%%%%%%%%%%%%%%%%%%%%%%%%%%%%%%%%%%%%%%%%%%%%%%%%%%%%%%%%%%%%%%%%%%%%%%%%

\section{Results} \label{Sec:Results}

%%%%%%%%%%%%%%%%%%%%%%%%%%%%%%%%%%%%%%%%%%%%%%%%%%%%%%%%%%%%%%%%%%%%%%%%%%%%%%%%%%%%%%%%%%%%%%%%%%%%%%%%%
\section{Discussion and Conclusions} \label{Sec:Discussion}

%====================================================================
\section*{Acknowledgments}
%====================================================================

\noindent JR acknowledges support from the Royal Society and Science Foundation Ireland under
grant number URF$\backslash$R1$\backslash$191132.

JR wishes to acknowledge the DJEI/DES/SFI/HEA Irish Centre for High-End Computing (ICHEC) for the
provision of computational facilities and support on which the zoom simulations were run.
The freely available plotting library {\sc
matplotlib} \citep{matplotlib} was used to construct numerous plots within this
paper. Computations and analysis described in this work were performed using the
publicly-available \enzo{}\citep{Enzo_2014, Enzo_2019} and \yt{} \citep{YT} codes,
which are the product of a collaborative effort of many independent scientists
from numerous institutions around the world. Their commitment to open science
has helped make this work possible.



\label{lastpage}
\bibliographystyle{mn2e}
\bibliography{mybib}
\end{document}


